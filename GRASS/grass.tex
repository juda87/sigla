%%%%%%%%%%%%%%%%%%%%%%%%%%%%%%%%%%%%%%%%%
% Programming/Coding Assignment
% LaTeX Template
%
% This template has been downloaded from:
% http://www.latextemplates.com
%
% Original author:
% Ted Pavlic (http://www.tedpavlic.com)
%
% Note:
% The \lipsum[#] commands throughout this template generate dummy text
% to fill the template out. These commands should all be removed when
% writing assignment content.
%
% This template uses a Perl script as an example snippet of code, most other
% languages are also usable. Configure them in the "CODE INCLUSION
% CONFIGURATION" section.
%
%%%%%%%%%%%%%%%%%%%%%%%%%%%%%%%%%%%%%%%%%
%----------------------------------------------------------------------------------------
% PACKAGES AND OTHER DOCUMENT CONFIGURATIONS
%----------------------------------------------------------------------------------------
\documentclass{article}
\usepackage{url}
\usepackage{fancyhdr} % Required for custom headers
\usepackage{lastpage} % Required to determine the last page for the footer
\usepackage{extramarks} % Required for headers and footers
\usepackage[usenames,dvipsnames]{color} % Required for custom colors
\usepackage{graphicx} % Required to insert images
\usepackage{listings} % Required for insertion of code
\usepackage{courier} % Required for the courier font
\usepackage{lipsum} % Used for inserting dummy 'Lorem ipsum' text into the template
\usepackage[utf8]{inputenc}
\usepackage{verbatim}
\usepackage{cite}
% Margins
\topmargin=-0.45in
\evensidemargin=0in
\oddsidemargin=0in
\textwidth=6.5in
\textheight=9.0in
\headsep=0.25in
% Set up the header and footer
\pagestyle{fancy}
\lhead{\Curso} % Top left header
\chead{\firstxmark} % Top center head
\rhead{\Titulo} % Top right header
\lfoot{\SIGLA} % Bottom left footer
\cfoot{\lastxmark} % Bottom center footer
\rfoot{Pagina\ \thepage\ de\ \protect\pageref{LastPage}} % Bottom right footer
\renewcommand\headrulewidth{0.4pt} % Size of the header rule
\renewcommand\footrulewidth{0.4pt} % Size of the footer rule
\setlength\parindent{0pt} % Removes all indentation from paragraphs
%----------------------------------------------------------------------------------------
% DATOS DEL DOCUMENTO
%----------------------------------------------------------------------------------------
\newcommand{\Titulo}{Geographic Resources Analysis Support System: Primeros pasos} % Titulo del Tema
\newcommand{\Fecha}{Miércoles,\ Septiembre\ 3,\ 2014} % Fecha de creacion
\newcommand{\Curso}{GRASS} % Curso/clase
\newcommand{\Profesion}{Ing.Ambiental} % Espacio para colocar hora
\newcommand{\SIGLA}{SIGLA} % autor / pero se usa para poner el nombre del capitulo quedo como bien.
\newcommand{\Autor}{Juan David Rondón Díaz} % Nombre de lautor
%----------------------------------------------------------------------------------------
% TITLE PAGE
%----------------------------------------------------------------------------------------
\title{
\vspace{2in}
\textmd{\textbf{\Curso:\ \Titulo}}\\
\normalsize\vspace{0.1in}\small{Creado\ en\ \Fecha}\\
\vspace{0.1in}\large{\textit{\Autor\ \\ \Profesion}}
\vspace{3in}
}
\author{\textbf{\SIGLA}}
\date{\today} % Insert date here if you want it to appear below your name
%----------------------------------------------------------------------------------------
\begin{document}
\maketitle
%----------------------------------------------------------------------------------------
% TABLE OF CONTENTS
%----------------------------------------------------------------------------------------
%\setcounter{tocdepth}{1} % Uncomment this line if you don't want subsections listed in the ToC
\newpage
\tableofcontents
\newpage
\section{Descipción general}
\subsection{¿Que es GRASS?}
Geographic Reources Analysis Suppor Sustem, normalmente conocido como GRASS GIS, es un Sistema de Información Geográfica (SIG) usado para la administración de datos, procesamiento de imágenes, producción de gráficos, modelación espacial y visualización de varios tipos de datos. Es un programa libre (Open Source) y se encuentra bajo la licencia \textit{GNU General Public License (GPL)>=  2}. GRASS GIS es un proyecto oficial de la \textit{Open Source Geospatial Foundation}.

\subsection{Características generales de GRASS GIS}
GRASS GIS contienes cerca de 350 módulos para desplegar mapas e imágenes el montior así como en el papel; manipular datos raster y vector incluyendo \textbf{\textit{redes}} vector; procesamiento de datos de im\'agenes multiespectrales; y crear, administrar y almacenar datos espaciales. GRASS GIS ofrece una interfaz gráfica de usuario así como una sintaxis de línea de comandos para facilitar las operaciones. GRASS GIS sirve de interface con impresoras, plotters, digitalizadores, y bases de datos tanto para la generación de nuevos datos, así como la administración de información existente.

\subsection{Capacidades}
\begin{itemize}
\item Análisis Raster
\item Análisis 3D-Raster (Voxel)
\item Análisis Vector
\item Análsis de datos puntuales
\item Procesamiento de imágenes
\item Análisis de MDT
\item Geocoding
\item Visualización
\item Creación de mapas
\item Soporte SQL
\item Geoestadística
\end{itemize}

\section{Instalación}
\subsection{GNU/Linux}
A continuación se explica de manera breve la forma de instalación para algunas ditros de Linux.

\subsubsection{Ubuntu}
Ya que los binarios se encuentran disponibles de apt-synaptic. Simplemente desde el terminal es necesario ejecutar:

\begin{lstlisting}[language=bash, frame=single]
sudo apt-get install grass grass-doc
\end{lstlisting}

\subsubsection{Fedora}
Gracias a Fedoraproject se tienen paquetes-RPM de versiones estables ya listas para instalar en sistemas Fedora:

\begin{lstlisting}[language=bash, frame=single]
yum install grass
\end{lstlisting}

\subsubsection{OpenSUSE}

En primer lugar es necesario añadir el repositorio GEO que contiene los paquetes-RPM de GRASS entre otras herramientas del campo de la geomática, luego se refrescan los listados de repositorios en el computador y con esto ya se puede instalar. Para ello se debe ejecutar en la terminal:

\begin{lstlisting}[language=bash, frame=single]
sudo zypper ar \
http://download.opensuse.org/repositories/Application:/Geo/openSUSE_12.3/ GEO
sudo zypper refresh
sudo zypper install grass
\end{lstlisting}

Es importante tener en cuenta que las instrucciones del cajón anterior resultan apropiadas para la versión 12.3 de OpenSUSE, si se tiene una versión diferente es necesario cambiar este numero en la instrucción por la que corresponda a la versión instalada.

%----------------------------------------------------------------------------------------
\end{document}
